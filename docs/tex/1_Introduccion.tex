\capitulo{1}{Introducción}

\section{Contenido del trabajo}

Llegado al último curso todos los alumnos de la UBU deben realizar su Trabajo de Fin de Grado (TFG) y son muchas las dudas que tienen acerca del funcionamiento del mismo, especialmente en la enseñanza online en la que al no haber un contacto tan estrecho con profesores y otros alumnos se tiene menos información. Motivo por el que los alumnos, aparte de consultar la extensa documentación aportada en las distintas plataformas de la UBU consultan a los profesores por email para despejar sus dudas.

Muchas de estas preguntas acerca del trabajo ya han sido respondidas anteriormente, o están explicadas en alguna parte de la documentación de UBUVirtual o de la página del grado. Con el propósito de ayudar a los alumnos a despejar estas dudas y reducir la carga de preguntas que reciben los profesores, se desarrolla este chatbot cuya finalidad es la de dar respuesta de forma automática a las preguntas de los usuarios acerca de todos los aspectos relacionados con la asignatura. El chatbot está integrado en la plataforma de UBUVirtual de la asignatura de TFG, a la que todos los alumnos matriculados en el trabajo tienen acceso. 

Por medio del aprendizaje automático nuestro chatbot será capaz de entender y responder las preguntas que los alumnos le formulen, y a su vez aprenderá de aquellas que no las haya recibido anteriormente para ser capaz de responderlas en la próxima ocasión en que se le pregunten.\\
Con ello resolveremos la doble problemática para profesores y alumnos, con la ventaja añadida de que el bot responderá ininterrumpidamente a cualquier hora del día.



\section{Estructura de la memoria}


\section{Materiales adjuntos}

