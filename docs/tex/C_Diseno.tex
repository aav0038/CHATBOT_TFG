\apendice{Especificación de diseño}

\section{Introducción}

En este apartado se explica con detalle como se han estructurado los datos, los procedimientos y la arquitectura que se ha utilizado.

\section{Diseño de datos}

Los datos se estructuran en torno a los \textit{intents} y \textit{entities}.

\subsection{Entities}

Hemos considerado las tres entidades mostradas en la figura \ref{fig:Entities}, al darse el caso de que había sinónimos de estos que se repetían en muchas ocasiones.

\imagenEnana{Entities}{Entidades utilizadas en el agente.}

\subsection{Intents}

Aquí es donde almacenamos nuestras frases de entrenamiento y respuestas. Disponemos de un total de 205 archivos JSON para 103 intents diferentes en la modalidad online, y 195 archivos para 98 \textit{intents} diferentes en la modalidad presencial. En las Figuras \ref{fig:bbdd1} y \ref{fig:bbdd2} se presenta el esquema utilizado para la versión online.

\imagenGrande{bbdd1}{Diseño de \textit{intents} del agente para la modalidad online (Parte 1).}

\imagenGrande{bbdd2}{Diseño de \textit{intents} del agente para la modalidad online (Parte 2).}

Se ha utilizado la nomenclatura de comenzar el nombre del \textit{intent} con un simbolo \textit{\_} para determinar aquellos cuya respuesta es una \textit{rich response}.

\newpage
Para hacer legible y mantenible nuestra base de datos de preguntas y respuestas se dispone de un documento de texto en el que se recopila toda la información. De esta manera una persona ajena al proyecto podrá comprender la estructura rápidamente.

\imagenGrande{recopilacion}{Primera página del documento recopilatorio de preguntas y respuestas.}


\section{Diseño procedimental}

El proceso de interaccion es siempre el mismo: el usuario formula una pregunta desde el \textit{chatbot} de UBUVirtual o Slack, esta pregunta se envía al agente de Dialogflow que está alojado en un servidor. 

A continuación, este mensaje se envía a la API Natural Language de Google Cloud, la cuál mediante un algoritmo asigna un \textit{intent} de salida.

En el agente de Dialogflow se consulta la respuesta del \textit{intent} asignado y se devuelve al usuario por medio de la interfaz por la que se esta operando.

\imagen{diagramaSecuencia}{Diagrama de secuencia.}

\newpage

\section{Diseño arquitectónico}

El agente de Dialogflow está alojado en un servidor de Google Cloud. Dicho servidor reside en Estados Unidos, y Dialogflow nos lo proporciona de manera gratuita al utilizar una versión reducida.

El usuario accede por medio de su navegador al \textit{chatbot} de la asignatura, el cuál está integrado en UBUVirtual. Es decir, en primer lugar accede al servidor en el que se aloja la plataforma Moodle de la Universidad. 

La arquitectura que sigue la interacción es la del modelo cliente-servidor: el usuario introduce una pregunta en el \textit{chatbot} (cliente) la cuál envía una petición al servidor (agente de Dialogflow) que
 la procesa y devuelve una respuesta al cliente.

\imagen{diagramaDespliegue}{Diagrama de despliegue.}


