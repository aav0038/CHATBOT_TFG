\apendice{Plan de Proyecto Software}

\section{Introducción}

Para el correcto desarrollo del proyecto es imprescindible establecer una planificación adecuada. En esta planificación se evalúa la viabilidad de la solución propuesta, y una vez se ha determinado que es viable se establecen una serie de etapas que va a seguir la implementación. También se establecen el calendario con los plazos para cada una de estas etapas.

Este Plan de Proyecto Software se divide en dos apartados:

\begin{itemize}
	\tightlist
	\item
	\textbf{Planificación temporal:} establece los plazos de desarrollo del proyecto. Se establece una fecha de inicio y fin de proyecto, así como el intervalo de fechas previstas para cada desarrollo incremental. Las fechas marcadas han sufrido adaptaciones a lo largo del desarrollo y las cargas de trabajo de los intervalos no son semejantes. La disponibilidad del alumno ha marcado el volumen de trabajo de cada una de las fases.
	\newline
	\item
	\textbf{Estudio de viabilidad:} antes de comenzar a desarrollar el proyecto es necesario determinar si es viable.
	\begin{itemize}
		\tightlist
		\item
		\textbf{Viabilidad económica:} realizar una estimación de los posibles costes y beneficios. Se analizan todo tipo de costes: humanos, software, hardware, instalaciones...
		\item
		\textbf{Viabilidad legal:} determinar si este proyecto cumple todas las leyes y regulaciones.
	\end{itemize}
\end{itemize} 



\newpage
\section{Planificación temporal}

En el desarrollo del proyecto se siguió la metodología ágil Scrum \cite{Scrum}, estudiada a lo largo de la carrera.
No se aplicó de manera estricta esta metodología, ya que se dieron varios factores que no lo hicieron posible:

	\begin{itemize}
	\tightlist
	\item
	No había un equipo de trabajo como tal, únicamente el alumno y tutor. En ocasiones puntuales colaboró un tercero: el presidente del tribunal. 
	\item
	 Al compaginar estudio y trabajo los \textit{sprints} en algunas ocasiones no fueron tan breves como se hubiera deseado, aunque nunca excedieron el máximo de un mes que establece esta metodología.
	\end{itemize}

De la misma manera, se puede determinar que se utilizo esta metodología basándonos en los siguientes puntos:
	\begin{itemize}
	\tightlist
	\item
	Desarrollo incremental del producto software por medio de \textit{sprints}.
	\item
	Duración general de dos semanas para los \text{sprints}.
	\item
	Reuniones entre alumno y tutor al final de cada \textit{sprint}, en las que se analizaba el trabajo realizado y se establecían los objetivos del siguiente \text{sprint}.
	\end{itemize}


\subsection{Sprint 0 (23/10/20 - 06/11/20)}

El inicio de este \textit{sprint} lo marcó la primera reunión con Raúl Marticorena, en la que se dieron las indicaciones de lo qué se buscaba con el proyecto y se establecieron los primeros objetivos.\\
Se estableció LaTeX como herramienta a utilizar para la documentación de la memoria y anexos.\\
Las primeras tareas fueron: estudiar las FAQ de la asignatura a incorporar en la base de datos del \textit{chatbot} y determinar que herramienta se iba a utilizar. 


\subsection{Sprint 1 (07/11/20 - 20/11/20)}

Se establece Dialogflow como motor NLP elegido y se crea el proyecto. 
Se crea un documento en el que ir recopilando las preguntas y respuestas de manera conjunta con Carlos López. 
En la reunión de final de \textit{sprint} se estudian las herramientas disponibles para trabajar con LaTeX.\\
Como objetivos se fijan analizar las posibles integraciones y estudiar la viabilidad de añadir hipervínculos en las respuestas del \textit{chatbot}.

\subsection{Sprint 2 (21/11/20 - 27/11/20)}

Como metas se establece el trabajo en la memoria, estudiando los trabajos relacionados y la descarga de un Moodle local para probar la integración del \textit{chatbot}.

\subsection{Sprint 3 (28/11/20 - 11/12/20)}

Se trabajó en la integración con UBUVirtual, y en la reunión del final del \textit{sprint} se probó conjuntamente a integrarlo, de manera que quedase oculto a todos los usuarios. 
Se logró dicha integración de una primera versión muy básica del \textit{chatbot}, sin ningún tipo de personalización CSS ni soporte de hipervínculos.

\subsection{Sprint 4 (12/12/20 - 15/01/21)}
\subsection{Sprint 5 (16/01/21 - 29/01/21)}

\subsection{Sprint 6 (30/01/21 - 05/03/21)}

\subsection{Sprint 7 (06/03/21 - 02/04/21)}

\subsection{Sprint 8 (03/04/21 - 07/05/21)}

Este sprint tuvo una duración de más de dos meses, esto se decidió así por parte del alumno ante la falta de tiempo generada por su trabajo, exámenes y prácticas de la Universidad, y prueba final de estudios de ruso.

En este periodo se continuó ampliando la base de datos del NLP realizando revisiones periódicas de los \textit{logs} y añadiendo las nuevas preguntas sin respuesta al documento para que Raúl Marticorena en colaboración con Carlos López les diesen respuesta.

Se realiza un estudio numérico en el que todas las conversaciones son analizadas, y se determina para cada pregunta de los usuarios si se contesta de manera correcta, incorrecta, o si la pregunta era incoherente. Por medio de Excel se realiza un estudio numérico y gráfico de los resultados obtenidos.
En la reunión se comenta este gráfico analizando los resultados obtenidos hasta la fecha, se comenta la situación del proyecto y las tareas restantes, que giraban principalmente en torno a la documentación.

\subsection{Sprint 9 (08/05/21 - 21/05/21)}

\subsection{Sprint 10 (22/05/21 - 09/06/21)}

\subsection{Sprint 11 (10/06/21 - 18/06/21)}

En esta semana se dedica la mayor parte del tiempo a la documentación de la memoria. Se añaden nuevas preguntas a la base de datos del NLP generadas a partir de los \textit{logs}.\\
También se genera e integra el logo del proyecto en el \textit{chatbot} de UBUVirtual.
En la reunión se comentan algunos problemas encontrados en la integración de Slack y se fija como objetivo arreglarlos.

\subsection{Sprint 12 (19/06/21 - 24/06/21)}

Los esfuerzos se centraron en la elaboración de la memoria y anexos. Se finaliza la implementación de las respuestas en la versión de Slack y se actualizan las de UBUVirtual para ambas modalidades. Con esta última actualización se da por cerrado los incrementos en la base de datos del \textit{NLP}. También se actualiza el estudio numérico con las últimas conversaciones registradas.


\newpage
\section{Estudio de viabilidad}

\subsection{Viabilidad económica}

\subsection{Viabilidad legal}


