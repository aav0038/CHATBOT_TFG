\capitulo{6}{Trabajos relacionados}

El primer bot conversacional fue creado entre 1964 y 1966 por Joseph Weizenbaum en el Laboratorio de Ciencias de la Computación e Inteligencia Artificial del Instituto de Tecnología de Massachusetts \cite{wiki:eliza}. Simulaba ser un psicólogo y se creó con el propósito de demostrar la superficialidad de las conversaciones entre humanos y máquinas.

Desde entonces han evolucionado, especialmente en los últimos años y aún más tras la pandemia de Covid-19, reflejo de ello es un estudio de \textit{Markets And Markets} \cite{ChatbotGrowth} que apunta un incremento previsto en el volumen de negocio de los chatbots de 4.8 billones de dólares en 2020 a 13.9 billones en 2025.

Las finalidades de los chatbots son muy variadas: responder preguntas frecuentes, ofrecer soporte al cliente, encuestas, reservas, etc. En este apartado vamos a analizar algunos trabajos relacionados. Al tratarse de un campo tan amplio vamos a focalizar la búsqueda con una serie de criterios. El primer criterio es que únicamente vamos a tener en cuenta los chatbots programados en castellano, ya que el idioma es un aspecto relevante al ser en torno a lo que giran los NLP.
El segundo criterio va a ser el del fin; solo vamos a buscar chatbots implementados en universidades españolas con el fin de responder dudas frecuentes de estudiantes, siendo este el mismo propósito que el de nuestro proyecto.
\\No obstante, en primer lugar vamos a analizar los proyectos de la Universidad de Burgos que tengan alguna relación con los chatbots, para tener una visión de la situación actual de la UBU en esta tecnología.

\newpage


\section{Proyectos de la Universidad de Burgos}

\subsection{\emph{UBUAssistant}}

\imagenPequena{UBUAssistant}{UBUAssistant logo}

Proyecto de Daniel Santidrián Alonso continuado por Carlos González Calatrava en 2018 tutorizado por Pedro Renedo Fernández.
Se trata de un asistente virtual para Android que cuenta con asistente de voz. El objetivo es facilitar las búsquedas en la web de la UBU al realizarlas desde un smartphone.
La aplicación está alojada en un servidor de Azure que ofrece una suscripción gratuita para estudiantes, aunque con ciertas limitaciones. Utiliza el algoritmo de Razonamiento Basado en Casos. \cite{UBUAssistant}


\subsection{\emph{Chatbot for Tourist Recommendations}}

\imagenPequena{touristBot}{Tourist Bot logo}
Proyecto de Jasmin Wellnitz en 2017 tutorizado por el Dr. Bruno Baruque Zanon. 
Chatbot implementado en Telegram actualmente fuera de servicio cuya finalidad era dar recomendaciones turísticas en inglés.
Utilizó la misma tecnología que este proyecto, en aquel entonces se llamaba API.AI y era el mismo NLP que es ahora Dialogflow tras ser comprado en 2016 por Google que optó al año siguiente por cambiarle el nombre.
El proyecto esta alojado en la \textit{Platform as a Service} Heroku. Utiliza la versión de prueba gratuita pero tiene limitaciones en cuanto al uso mensual. \cite{ChatbotTourist}


\subsection{\emph{UBUVoiceAssistant}}

\imagenPequena{UBUVoiceAssistant}{UBUVoiceAssistant logo}

Proyecto de Álvaro Delgado Pascual tutorizado por el Dr. Raúl Marticorena Sanchez.
Se trata de una aplicación que por medio de un asistente de voz permite al usuario obtener información sobre una plataforma de Moodle. Está desarrollado en Python y utiliza el asistente de voz Mycroft. \cite{UBUVoiceAssistant}

\newpage
\section{Proyectos externos}

\subsection{\emph{Chatbot Lola}}

\textit{Lola} es un chatbot de la Universidad de Murcia creado en 2018 cuya finalidad es la de resolver dudas a nuevos estudiantes acerca del proceso de admisión y matrícula.\\
Lola es un proyecto del Servicio de  Información Universitario (SIU) que se ha desarrollado con el Área de Tecnologías de la Información y de las Comunicaciones Aplicadas de la Universidad de Murcia (ATICA) y la empresa 1 million bot. \cite{Lola}\\

\imagenMediana{Lola}{Logo y funcionalidades de Lola}

Este mismo chatbot está implementado por la misma empresa en otras universidades: Universidad de Zaragoza, Complutense de Madrid y Universidad Politécnica de Valencia entre otras. Todos tienen el mismo funcionamiento aunque utilizan nombres comerciales distintos, siendo siempre un nombre de mujer.