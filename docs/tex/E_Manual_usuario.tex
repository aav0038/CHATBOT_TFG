\apendice{Documentación de usuario}

\section{Introducción}

En este apartado se explican los requerimientos de la aplicación para ser ejecutada tanto en la plataforma Moodle como en Slack. Se indica también el proceso de instalación en ambas plataformas. Finalmente, en el manual de usuario se dan las indicaciones para utilizar correctamente la aplicación.

\section{Requisitos de usuarios}

En esta sección se indican los requisitos para utilizar el chatbot en cada plataforma para la versión de la modalidad online.

\subsection{Plataforma UBU Virtual}

\begin{itemize}
	\tightlist
	\item
	Conexión a Internet.
	\item
	Navegador web en cualquier dispositivo: PC, Smartphone, Tablet, etc. \\
	Compatible con Chrome, Safari, Firefox, Edge y Opera entre otros. Incompatible con Internet Explorer.
	\item
	Cuenta activa en la plataforma UBUVirtual.
	\item
	Permiso de acceso al curso `TRABAJO FIN DE GRADO (Grupo 90) On-line' (id 11707).
	
\end{itemize}

\newpage
\subsection{Slack}	

\begin{itemize}
	\tightlist
	\item
	Conexión a Internet.
	\item
	Navegador web en cualquier dispositivo: PC, Smartphone, Tablet, etc. \\
	Compatible con Chrome, Safari, Firefox, Edge y Opera. \cite{SlackRequisitos}
	\item
	Cuenta de Slack y permiso de acceso al espacio de trabajo ubuchatbot.slack.com. 

\end{itemize}
 
 \newpage
\section{Instalación}

\subsection{Plataforma UBUVirtual}

No es necesario ningún proceso de instalación. Únicamente necesitaremos acceder a la asignatura en la plataforma UBUVirtual con un navegador compatible y tener JavaScript activado.\\

El enlace de acceso al chatbot es el siguiente: \url{https://ubuvirtual.ubu.es/mod/page/view.php?id=3132682}

\subsection{Slack}

Se puede acceder a esta plataforma mediante la versión web o la aplicación de escritorio.\\
En caso de que quisiéramos obtener la versión de escritorio, deberíamos acceder a \url{https://slack.com/intl/es-es/downloads} y seleccionar la versión compatible con nuestro sistema operativo, pero este paso no es necesario.

Como paso previo necesitaremos una invitación al espacio de trabajo que nos deberá haber enviado alguno de los miembros. En caso de que no se te haya facilitado esta invitación, escribe un email al administrador a su correo  aav0038@alu.ubu.es indicando tu correo de Google y te la hará llegar. \\

Te llegará un email a la bandeja de entrada con el título `Alfredo te ha invitado a trabajar en Slack'. En caso de no recibirlo se debe comprobar la bandeja de correo no deseado. 

\imagenMediana{Slack-EmailConfirmacion}{Email de confirmación una vez recibido acceso}

Para acceder a la aplicación, lo haremos por medio del botón morado de `Únete ahora' disponible en el email de confirmación como se muestra en la Figura \ref{fig:Slack-EmailConfirmacion}. \\
Nos llevará a una pantalla de login en la que al haber introducido un correo de Google, haremos clic en `Continuar con Google' como se muestra en la Figura \ref{fig:Slack-Unirse}.

\imagenMediana{Slack-Unirse}{Pantalla para unirse al espacio de trabajo de Slack}

Seleccionaremos nuestra cuenta de Google y haremos clic en el botón morado `Crear cuenta' como se muestra en la Figura \ref{fig:Slack-Unirse2}.

\imagenMediana{Slack-Unirse2}{Confirmación para crear la cuenta y acceder al espacio de trabajo.}

Tras unos segundos de carga habremos accedido al espacio de trabajo `UBU Chatbot'.

Alternativamente y en posteriores conexiones, se puede acceder a través del siguiente enlace que nos llevará al espacio de trabajo en el que está integrado el chatbot:\\
\url{https://ubuchatbot.slack.com}
\imagen{anexo-E3-1}{Espacio de trabajo de Slack}

Se nos pedirá credenciales para acceder de la manera que se muestra en la Figura \ref{fig:anexo-E3-1} y lo haremos de la  forma anteriormente explicada.

\newpage

\section{Manual del usuario}

\subsection{Plataforma UBU Virtual}

\subsubsection{Desde PC} \label{subsection:marca-manualpc}

Para la versión de ordenador, una vez hemos accedido a la página de UBUVirtual donde está disponible el chatbot, veremos como en la parte inferior derecha aparece un icono redondo con el logo del proyecto: un oso con un bocadillo de texto sobre un fondo granate. A los segundos aparece un mensaje de bienvenida junto a este, tal y como se muestra en la Figura \ref{fig:anexo-E4-1}.

\imagen{anexo-E4-1}{Visualización del chatbot inicialmente minimizado.}

Para abrir el chatbot haremos clic en el círculo del logo o en el mensaje de bienvenida. Se nos desplegará el chat, en el que aparecerá el mensaje de bienvenida y se habilitará una entrada de texto en la que deberemos hacer clic para empezar a introducir nuestros mensajes. Ver Figura \ref{fig:anexo-E4-2}.

\imagen{anexo-E4-2}{Ventana del chatbot desplegada.}

Podemos comenzar realizando cualquier pregunta. Es importante tener en cuenta que el chatbot esta diseñado para dar respuesta a \textbf{estudiantes online del TFG}. Es decir, si le hacemos preguntas como si fuésemos profesores de la asignatura no las va a saber responder, así como si le hacemos preguntas sobre otras asignaturas. Además, todas las respuestas las va a dar enfocadas a la modalidad online, que pueden no coincidir en muchos casos con la que se daría para presencial.\\
Comenzamos por ejemplo saludando al chatbot, aunque es un paso innecesario.

\imagenPequena{anexo-E4-3}{Respuesta a saludo.}

Como vemos en la Figura \ref{fig:anexo-E4-3} al no haber introducido una pregunta nos devolverá el mismo saludo inicial y nos volverá a pedir que le introduzcamos una pregunta.\\
En este caso vamos a formular la primera pregunta. Podemos hacer cualquier pregunta relativa al Trabajo de Fin de Grado. 

\imagenPequena{anexo-E4-4}{Pregunta bien formulada.}

En este primer ejemplo mostrado en la Figura \ref{fig:anexo-E4-4} se ha formulado la pregunta con signos de interrogación y sin fallos ortográficos. Esto no es un requisito necesario para que el chatbot sea capaz de responder. Podemos formular las preguntas de muchas maneras diferentes.


\imagenPequena{anexo-E4-5}{Distintas formulaciones.}

Como se ve en la Figura \ref{fig:anexo-E4-5}, distintas formulaciones de la misma pregunta han producido la misma respuesta, ya que el chatbot ha sido capaz de comprenderlas. \\
No obstante, hay que seguir una serie de buenas prácticas para que el chatbot funcione de la mejor manera posible:
\begin{itemize}
	\tightlist
	\item
	Solo una pregunta por mensaje.
	\item
	El mensaje ha de ser lo más concreto posible.
	\item
	La pregunta no debe dar lugar a ambigüedades. Ejemplo: si escribimos ``memoria'' no le estamos dando información suficiente para saber qué información queremos conocer acerca de la memoria.
	\item
	No hacer preguntas personales. Si le preguntamos cuál es nuestra hora de defensa no va a saber responderlo. Hay que recordar que no estamos identificados (el chatbot no sabe qué alumno somos) y solo se dan respuestas generales a todos los alumnos.
\end{itemize}


\imagenPequena{anexo-E4-6}{Pregunta mal formulada.}

En la Figura \ref{fig:anexo-E4-6} vemos un ejemplo de pregunta mal formulada. Buscaba el mismo objetivo que las anteriores, pero en este caso el chatbot no ha sido capaz de identificar que información quería conocer el alumno. Esto se debe a que la pregunta no ha sido concreta y ha introducido información que no tenía nada que ver con la pregunta. En este caso la respuesta ha sido un mensaje de error en el que se aportan indicaciones para formular mejor la pregunta y aumentar así las probabilidades de éxito.

\imagenPequena{anexo-E4-7}{Cerrar chatbot.}

La conversación continuará indefinidamente. Cuando deseemos cerrar la conversación simplemente hacemos clic en el círculo con la `x' en la parte inferior del chatbot, señalizado con la flecha roja en la Figura \ref{fig:anexo-E4-7}.


\imagen{anexo-E4-8}{Chatbot cerrado.}

Al hacer clic en el botón de cerrar se volverá al estado inicial en el que la conversación no está desplegada. En este caso, el mensaje de bienvenida ya no aparecerá y únicamente se verá el círculo con el logo del proyecto con el que haciendo clic podremos retomar la conversación cuando queramos. Ver Figura \ref{fig:anexo-E4-8}\\
Si cambiamos de página o refrescamos se reiniciará la conversación, borrándose los mensajes anteriores.
 
\newpage
\subsubsection{Smartphone}

El funcionamiento en Smartphone es muy similar al de PC, primero abrimos la página en la que está integrado el chatbot. 

\imagenPequena{anexo-E4-9}{Página del chatbot desde Safari en iOS.}

En la Figura \ref{fig:anexo-E4-9} podemos ver como en esta versión también aparece un pequeño icono con el logo del proyecto, esta vez sin que se despliegue el mensaje de bienvenida. Hacemos clic en dicho botón.

\newpage

Se nos habrá desplegado una ventana de chat algo diferente a la de versión PC, ya que en este caso ocupa toda la pantalla. Ver Figura \ref{fig:anexo-E4-10}

\imagenPequena{anexo-E4-10}{Ventana del chatbot en Safari iOS.}

Hacemos clic en el cuadro de texto inferior con el texto sombreado ``Haz una pregunta...''.

\newpage

Al clicar nos aparecerá el teclado del teléfono móvil como se muestra en la Figura \ref{fig:anexo-E4-11}, y ya podremos teclear nuestra pregunta. Una vez la tengamos escrita hacemos clic en el botón `intro' del teclado. Este botón puede ser distinto en otros dispositivos móviles.

\imagenPequena{anexo-E4-11}{Página chatbot desde Safari en iOS.}

El funcionamiento y recomendaciones para realizar las preguntas es el mismo que el explicado en la versión de PC, por lo que es recomendable leer dicha sección.

\newpage

Una vez hemos hecho clic en `intro' y enviado nuestro mensaje recibiremos la respuesta del chatbot.
\imagenPequena{anexo-E4-12}{Respuesta del chatbot y botón de cierre de teclado desde Safari en iOS.}

El zoom de la pantalla puede haber sufrido modificaciones, por lo que se recomienda ajustarlo a nuestro gusto.\\
Para cerrar el chat deberemos primero hacer clic en el botón de `ok' marcado en el cuadro rojo en la Figura \ref{fig:anexo-E4-12}, lo cual cerrará el teclado.

\newpage

Una vez se nos ha cerrado el teclado nos aparecerá una fila de opciones en la parte inferior de la pantalla. Hacemos clic en la flecha de volver marcada en rojo en la Figura \ref{fig:anexo-E4-13} para salir definitivamente del chat y volver a la página anterior.

\imagenPequena{anexo-E4-13}{Volver a la página anterior desde Safari en iOS.}

Deslizando la pantalla hacia la izquierda también se podía haber hecho este proceso sin necesidad de cerrar previamente el teclado.\\
Nos devolverá a la página principal de la asignatura.


\newpage
\subsection{Slack}

Accedemos al espacio de trabajo en el que está integrado el chatbot como se ha explicado en el apartado de instalación.\\
URL: \url{https://ubuchatbot.slack.com}

\imagen{anexo-E4-14}{Página principal del espacio de trabajo de Slack.}

Hacemos clic dentro de la sección de `Aplicaciones' en `UBU Asistente Virtual', marcado en la Figura \ref{fig:anexo-E4-14} con un rectángulo rojo.

\newpage

En la página que se nos habrá abierto nos aparecerá el mensaje de bienvenida del chatbot igual que se muestra en la figura \ref{fig:anexo-E4-15}. Ya podemos empezar a introducir nuestras preguntas de la misma forma que en el chatbot de UBUVirtual. Es recomendable seguir las mismas recomendaciones que se han dado en la primera parte del apartado \ref{subsection:marca-manualpc} .

\imagen{anexo-E4-15}{Página del chatbot en Slack.}

